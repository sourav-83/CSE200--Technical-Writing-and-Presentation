\documentclass{article}
\usepackage{graphicx} 
\usepackage{float}
\usepackage{wrapfig}
\usepackage{caption}
\usepackage{subcaption}
\usepackage{multicol}
\usepackage{multirow}
\usepackage{hyperref}
\usepackage{amsmath}


\title{CSE200: Online-1 on \LaTeX}
\author{Sourav (2205083)}
\date{January 3, 2026}

\begin{document}

\maketitle

\section{Introduction}
Document preparation systems are essential for producing structured and pro
fessional technical documents. \textbf{LATEX} allows authors to focus on content
while maintaining consistent formatting. This article demonstrates formatted
text, lists, equations, tables, and figures using commands taught in class.
\subsection{Text Emphasis and Fonts}
This sentence contains \textbf{bold text}, \textit{italic text}, \emph{emphasized text,} and \underline{underlined text.}
Font sizes can also vary such as \large{Large text}, \small{small text}, and 
\Large{huge text.}
\section{List structures}
\subsection{Mixed and Nested Lists}
\begin{itemize}
    \item Primary Feature
    \item Secondary Features
    \begin{enumerate}
        \item Formatting control
        \item Mathematical typesetting
        \begin{itemize}
            \item[-] Inline math
            \item[-] Display math
        \end{itemize}
    \end{enumerate}
\end{itemize}
\textbf{Note} Lists can be customized
\section{Mathematical Modeling}
Mathematical expressions are often used to represent data relationships. Consider variables $x_i$, $y$, and $z^2$.
\subsection{Displayed Equations}

    
\begin{equation}
    y_i = \alpha x_j^2 + \beta x_j + \gamma
\end{equation}
\begin{align}
    T 
    &= \sum_{i=1}^{n} (x_i - \mu)^2 \nonumber \\
    &= \sum_{i=1}^{n} x_i^2 - 2\mu \sum_{i=1}^{n} x_i + n\mu ^2
\end{align}

\[
f(x_i) = 
\begin{cases}
\frac{x_i^2}{\sigma^2} & \text{if } x_i \geq 0 \\
\frac{|x_i|}{\sigma} & \text{if } x_i < 0
\end{cases}
\]
\section{Tabular Data Presentation}
% Please add the following required packages to your document preamble:
% \usepackage{multirow}
\begin{table}[h]
\centering
\begin{tabular}{|l|ll|}
\hline
\multirow{2}{*}{Module}   & \multicolumn{2}{l|}{Score}        \\ \cline{2-3} 
                          & \multicolumn{1}{l|}{Theory} & Lab \\ \hline
Module A                  & \multicolumn{1}{l|}{75}     & 80  \\ \hline
\multirow{2}{*}{Module B} & 88                          & 90  \\
                          & 85                          & 87  \\ \hline
Module C                  & \multicolumn{1}{l|}{70}     & 72  \\ \hline
\end{tabular}
\caption{Module-wise Performance Summary}
\end{table}
\section{Visual Comparison}
\subsection{Asymmetric Subfigure Layout}

\begin{figure}[H]
\centering


\begin{subfigure}{0.8\textwidth}
    \centering
    \includegraphics[width=\linewidth]{A.png}
    \caption{Main Diagram}
\end{subfigure}

\vspace{1em}

\begin{subfigure}{0.35\textwidth}
    \centering
    \includegraphics[width=\linewidth]{B.png}
    \caption{Detail B}
\end{subfigure}
\hfill
\begin{subfigure}{0.35\textwidth}
    \centering
    \includegraphics[width=\linewidth]{C.png}
    \caption{Detail C}
\end{subfigure}

\caption{Asymmetric layout with one dominant figure}
\end{figure}


    

\end{document}
